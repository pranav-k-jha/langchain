\documentclass[8pt]{beamer}

\usetheme[progressbar=frametitle]{metropolis}
\usecolortheme{dolphin}
\usepackage{graphicx}
\usepackage{booktabs}
\usepackage{lipsum}
\usepackage{xcolor}
\usepackage{tikz}
\usepackage{fontspec}
\usepackage{unicode-math}
\usepackage{ragged2e}
\usepackage{tcolorbox}
\usepackage{colortbl}
\usepackage{hyperref} 
\usepackage{listings}

\lstset{
  language=Python,
  basicstyle=\ttfamily\small,
  keywordstyle=\color{blue},
  commentstyle=\color{gray},
  stringstyle=\color{red},
  showstringspaces=false,
  breaklines=true
}


\usepackage[font=scriptsize]{caption} 

\hypersetup{
    colorlinks=true,
    linkcolor=westernpurple,
    urlcolor=westernpurple
}

\usetikzlibrary{shapes, shadows, positioning, arrows.meta}

% Modern sans-serif font
\setsansfont[Ligatures=TeX]{Fira Sans}

% UWO Purple Theme

\definecolor{darkpurple}{RGB}{55,20,110} \definecolor{lightpurple}{RGB}{180, 160, 210}
\definecolor{westernpurple}{RGB}{81,40,136}

\setbeamertemplate{title page}{
    \begin{tikzpicture}[remember picture,overlay]
        \fill[top color=darkpurple, bottom color=westernpurple] 
            (current page.north west) rectangle (current page.south east);

        \node[anchor=north west, white, font=\LARGE\bfseries] 
            at ([xshift=1cm,yshift=-1.5cm]current page.north west) {\inserttitle};
        \node[anchor=north west, white, font=\large, text width=10cm, align=left] 
  at ([xshift=1cm,yshift=-3cm]current page.north west) {\insertsubtitle};

        \node[anchor=north west, white] 
            at ([xshift=1cm,yshift=-4.2cm]current page.north west) {\insertauthor};
        \node[anchor=north west, white] 
            at ([xshift=1cm,yshift=-4.8cm]current page.north west) {Supervised by: Prof. Dr. Apurva Narayan};
        
        % This node now has text width and left align for wrapping
        \node[anchor=north west, white, text width=10cm, align=left] 
            at ([xshift=1cm,yshift=-5.4cm]current page.north west) {\insertinstitute};
        
        \node[anchor=north west, white] 
            at ([xshift=1cm,yshift=-8cm]current page.north west) {\insertdate};

        \node[anchor=south east] 
            at ([xshift=-1cm,yshift=1cm]current page.south east) 
            {\includegraphics[height=1cm]{logo.pdf}};
    \end{tikzpicture}
}
\setbeamertemplate{frametitle}{
    \nointerlineskip
    \begin{tikzpicture}[remember picture, overlay]
        % Full-width top bar
        \fill[westernpurple] 
            (current page.north west) rectangle
            ([yshift=-0.8cm]current page.north east);
        
        % Title text: regular (not bold), vertically centered
        \node[anchor=west, font=\large, text=white, align=left, minimum height=0.8cm] 
            at ([xshift=0.5cm, yshift=-0.4cm]current page.north west) 
            {\insertframetitle};
    \end{tikzpicture}
    \vspace{0.9cm} % Reserve space below the bar
}


% Footer line
\setbeamertemplate{footline}{
  \begin{tikzpicture}[remember picture,overlay]
    % Background bar
    \fill[westernpurple] (current page.south west) rectangle ++(\paperwidth, 0.4cm);
    
    % Author / Institute on the left
    \node[anchor=south west, font=\tiny\bfseries, text=white] 
        at ([xshift=0.3cm,yshift=0.05cm]current page.south west) 
        {\insertauthor \hspace{0.5em}|\hspace{0.5em} \insertinstitute};
    
    % Page number on the right
    \node[anchor=south east, font=\tiny, text=white] 
        at ([xshift=-0.3cm,yshift=0.05cm]current page.south east) 
        {\insertframenumber{} / \inserttotalframenumber};
  \end{tikzpicture}
}


\setbeamertemplate{background}{
    \begin{tikzpicture}[remember picture,overlay]
        \fill[white] (current page.north west) rectangle (current page.south east);
        \foreach \i in {1,...,20} {
            \node[circle, fill=lightpurple!15, minimum size=0.5cm] 
                at (rand*\paperwidth, rand*\paperheight) {};
        }
    \end{tikzpicture}
}

\setbeamertemplate{items}[circle]
\setbeamertemplate{sections/subsections in toc}[circle]
\setbeamercolor{itemize item}{fg=westernpurple}
\setbeamercolor{enumerate item}{fg=westernpurple}
\setbeamercolor{section in toc}{fg=darkpurple}

% Modern block design
\setbeamertemplate{blocks}[rounded][shadow=true]
\setbeamercolor{block title}{bg=westernpurple,fg=white}
\setbeamercolor{block body}{bg=lightpurple!10, fg=black}

\title[LangChain]{LangChain: Concepts and Ecosystem Overview}
\subtitle{Understanding Chains, Agents, Retrieval, and Deployment}
\author[Pranav Jha]{\textbf{Pranav Jha}}
\institute[Western University]{Intelligent Data Science Lab (IDSL) \\ Department of Computer Science \& ECE  \\ \textbf{Western University, Canada}}

\date{\today}

\setbeamertemplate{navigation symbols}{}


\begin{document}

\begin{frame}[plain]
    \titlepage
\end{frame}

\begin{frame}{LangChain Core Components}
  \begin{columns}[T]
    \column{0.5\textwidth}
    \begin{block}{Framework Architecture}
      \begin{itemize}
        \item \textcolor{westernpurple}{\textbf{Chains}}: Structured LLM workflows
        \item \textcolor{westernpurple}{\textbf{Agents}}: Autonomous tool usage
        \item \textcolor{westernpurple}{\textbf{Retrieval}}: Vector/document search
      \end{itemize}
    \end{block}
    
    \begin{block}{Development Tools}
      \begin{itemize}
        \item \textcolor{westernpurple}{\textbf{LangSmith}}: Debugging \& monitoring
        \item \textcolor{westernpurple}{\textbf{LangServe}}: API deployment
        \item \textcolor{westernpurple}{\textbf{LCEL}}: Pipeline language (parallelization, fallback, tracing, batching, streaming, async, composition)
      \end{itemize}
    \end{block}
    
    \column{0.5\textwidth}
    \begin{block}{Key Features}
      \begin{itemize}
        \item Model-agnostic framework
        \item Python/JavaScript support
        \item Modular component design
      \end{itemize}
    \end{block}
    
    \begin{tcolorbox}[colback=lightpurple!10,colframe=westernpurple,title=Why LangChain?]
      Enables building production-ready GenAI applications with:\\[5pt]
      \begin{itemize}
        \item Observable pipelines
        \item Scalable deployment
        \item Enterprise tooling
      \end{itemize}
    \end{tcolorbox}
  \end{columns}
\end{frame}

\begin{frame}{Implementation Roadmap}
  \begin{columns}[T]
    \column{0.5\textwidth}
    \begin{block}{Learning Path}
      \begin{enumerate}
        \item Ecosystem fundamentals
        \item Chains \& Agents
        \item Real-world projects
        \item Advanced tooling
        \item Production deployment
      \end{enumerate}
    \end{block}
    
    \begin{block}{Technical Stack}
      \begin{itemize}
        \item Model I/O management
        \item Reusable templates
        \item API integrations
        \item FastAPI backend
      \end{itemize}
    \end{block}
    
    \column{0.5\textwidth}
    \begin{block}{Key Benefits}
      \begin{itemize}
        \item Composable workflows
        \item Built-in observability
        \item End-to-end toolchain
      \end{itemize}
    \end{block}
    
    \begin{alertblock}{Applications}
      \begin{itemize}
        \item Enterprise chatbots
        \item Document processing
        \item AI assistants
        \item Custom LLM solutions
      \end{itemize}
    \end{alertblock}
    
    \vspace{0.5cm}
    \centering
    \textcolor{westernpurple}{\textbf{Next: Hands-on implementation examples}}
  \end{columns}
\end{frame}

\begin{frame}{Generative AI vs AI Agents vs Agentic AI}

\begin{itemize}
    \item \textbf{Generative AI:}
    \begin{itemize}
        \item Creates content (text, images, code, etc.)
        \item Focused on generation, not action or planning
        \item Stateless, reactive (e.g., GPT, DALL·E)
    \end{itemize}

    \vspace{0.3cm}
    \item \textbf{AI Agents:}
    \begin{itemize}
        \item Perceive, decide, and act in an environment
        \item Use tools, maintain limited state, plan short-term
        \item Goal-driven, but often task-specific
    \end{itemize}

    \vspace{0.3cm}
    \item \textbf{Agentic AI:}
    \begin{itemize}
        \item High autonomy, long-term planning and reflection
        \item Uses memory (episodic, vector) and adaptive strategies
        \item Persistent behavior and self-improvement
    \end{itemize}
\end{itemize}

\end{frame}

\begin{frame}{Comparison of AI Types}

\begin{tabular}{|l|c|c|c|}
\hline
\textbf{Feature} & \textbf{Generative AI} & \textbf{AI Agents} & \textbf{Agentic AI} \\
\hline
Output & Content & Decisions + Actions & Goal Completion \\
\hline
Autonomy & Low & Medium & High \\
\hline
Memory & None & Task-level & Persistent \\
\hline
Planning & Minimal & Task-oriented & Long-term, strategic \\
\hline
Tool Use & Rare & Common & Adaptive \\
\hline
\end{tabular}

\end{frame}

\begin{frame}{LangChain: Framework for LLM Applications}

\begin{columns}

% Left Column: Textual Content
\begin{column}{0.5\textwidth}
\begin{itemize}
    \item \textbf{LangChain} is a framework for building LLM-powered apps.
    \item Supports:
    \begin{itemize}
        \item LLMs
        \item Embedding models
        \item Vector stores
        \item Tool integrations
    \end{itemize}

    \item \textbf{Lifecycle Features:}
    \begin{itemize}
        \item \textbf{Development:} LangChain + LangGraph for stateful agents
        \item \textbf{Production:} LangSmith for monitoring + evaluation
        \item \textbf{Deployment:} LangGraph Platform for API-ready apps
    \end{itemize}
\end{itemize}
\end{column}

% Right Column: Image
\begin{column}{0.5\textwidth}
\centering
\includegraphics[width=1.1\textwidth]{langchain_intro.png} % Make sure to add the image file

\vspace{0.2cm}
\scriptsize{
Image Source: \href{https://python.langchain.com/docs/introduction/}{LangChain Documentation}
}
\end{column}

\end{columns}

\end{frame}

\begin{frame}[fragile]{Using Google Gemini with LangChain}

\begin{columns}

% Left Column: Steps
\begin{column}{0.45\textwidth}
\textbf{Steps to use Gemini (Google Generative AI):}
\vspace{0.2cm}

\begin{enumerate}
    \item Set your Google API key securely
    \item Import LangChain's Gemini class
    \item Initialize the model
    \item Send a prompt using \texttt{invoke()}
\end{enumerate}

\vspace{0.2cm}
\scriptsize{
API Key: \href{https://makersuite.google.com/app/apikey}{makersuite.google.com/app/apikey}
}
\end{column}

% Right Column: Code
\begin{column}{0.55\textwidth}
\textbf{Example Code:}
\vspace{0.2cm}

\begin{lstlisting}[language=Python]
import getpass, os
from langchain.chat_models import ChatGoogleGenerativeAI

if not os.environ.get("GOOGLE_API_KEY"):
    os.environ["GOOGLE_API_KEY"] = getpass.getpass(
        "Enter API key for Google Gemini: "
    )

model = ChatGoogleGenerativeAI(
    model="gemini-1.5-flash"
)
response = model.invoke("Hello, world!")
print(response)
\end{lstlisting}
\end{column}

\end{columns}

\end{frame}


\begin{frame}{Retrieval-Augmented Generation (RAG)}

\begin{itemize}
    \item \textbf{Problem:} LLMs have limited context windows and outdated training data.
    \vspace{0.2cm}
    
    \item \textbf{Solution:} \textbf{RAG} enhances LLMs by retrieving relevant documents from an external knowledge base at inference time.
    \vspace{0.2cm}
    
    \item \textbf{Architecture:}
    \begin{itemize}
        \item \textbf{Retriever:} Finds relevant documents using vector similarity (e.g., FAISS, Chroma).
        \item \textbf{Generator:} An LLM (e.g., GPT, Mistral) takes the retrieved info as context and generates an answer.
    \end{itemize}
    \vspace{0.2cm}
    
    \item \textbf{Benefits:}
    \begin{itemize}
        \item Dynamic access to up-to-date or domain-specific data
        \item Reduces hallucinations by grounding output in real documents
    \end{itemize}
    
    \item \textbf{Use Cases:}
    \begin{itemize}
        \item Internal enterprise chatbots
        \item Legal, medical, and technical assistants
        \item Customer support over large knowledge bases
    \end{itemize}

    
\end{itemize}

\end{frame}

\begin{frame}{How RAG Works (Diagram)}

\centering
\includegraphics[width=0.9\textwidth]{rag_diagram.jpg}

\vspace{0.4cm}
\scriptsize{
Source: \href{https://aws.amazon.com/what-is/retrieval-augmented-generation/}{aws.amazon.com/what-is/retrieval-augmented-generation}
}

\end{frame}

\begin{frame}{Key Takeaways \& Next Steps}
  \begin{columns}[T]
    \column{0.5\textwidth}
    \begin{block}{What We Covered}
      \begin{itemize}
        \item LangChain ecosystem overview
        \item Core components: Chains, Agents, Retrieval
        \item Development tools: LangSmith, LangServe, LCEL
        \item RAG implementation with Gemini
        \item Model-agnostic framework benefits
        \item AI types comparison
      \end{itemize}
    \end{block}
    
    \column{0.5\textwidth}
    \begin{block}{Next Steps}
      \begin{itemize}
        \item Hands-on implementation
        \item Real-world project building
        \item Advanced tooling exploration
        \item Production deployment
      \end{itemize}
    \end{block}
    
    \begin{tcolorbox}[colback=lightpurple!10,colframe=westernpurple,title=Remember]
      LangChain provides the complete ecosystem for building production-ready GenAI applications
    \end{tcolorbox}
  \end{columns}
\end{frame}

\begin{frame}{LangChain Ecosystem Overview}
  \begin{columns}[T]
    \column{0.6\textwidth}
    \begin{block}{Core Ecosystem Components}
      \begin{itemize}
        \item \textbf{Observability}: LangSmith for debugging, evaluation, annotation
        \item \textbf{Deployment}: LangServe for REST API creation
        \item \textbf{Templates}: Reusable application blueprints
        \item \textbf{Integration}: Model I/O, retrieval, agents, tools
        \item \textbf{Protocol}: LCEL for composition and workflows
      \end{itemize}
    \end{block}
    
    \column{0.4\textwidth}
    \centering
    \includegraphics[width=\textwidth]{langchain_ecosystem.png}
    \scriptsize{Source: LangChain Documentation}
  \end{columns}
\end{frame}

\end{document}